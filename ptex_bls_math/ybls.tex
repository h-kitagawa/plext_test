\ifdefined\enablecjktoken
  \documentclass[uplatex]{jsarticle}
\else
  \documentclass{jsarticle}
\fi
\usepackage{amsmath}
\textwidth300pt

\begin{document}
\noautoxspacing
\showboxdepth10000
\showboxbreadth10000
\tracingonline1
\def\R{\vrule height 0.4pt depth 0pt width 10pt}
\def\cs#1{\texttt{\char92 #1}}
new primitives
\begin{itemize}
  \item \cs{textbaselineshiftfactor} (1000)
  \item \cs{textbaselineshiftfactor} (700)
  \item \cs{textbaselineshiftfactor} (500)
\end{itemize}
「過剰なベースライン補正」を打ち消す(=欧文位置を揃える)ための設定値.
例:\cs{scriptstyle}では,明示的に挿入されたボックスは
\cs{ybaselineshift}の$700/1000=0.7$倍だけ上昇.

\makeatletter
\let\orig@adjustbaseline=\adjustbaseline
\def\adjustbaseline{\orig@adjustbaseline\baselineskip=1.6zh\ybaselineshift=1zh}
\normalsize\adjustbaseline

あいうえおABC$%
  \mathop{\hbox{漢X\R}}\vbox{\hsize=3em 字Y\hss}\R xyz\frac{\text{\adjustbaseline 漢X\R}\R213}{s}
  \displaystyle\int_{0}^{\text{\adjustbaseline 漢X\R}\R1}%\showlists
$\vrule height 40pt depth 30pt

\makeatletter\let\underline=\@@underline
%ABCあいう\underline{\hbox{漢X\R}\R213}
ABCあ$\underline{\hbox{漢X}213}$%
ABCあ$\underline{\hbox{漢X}}$\vrule height 40pt depth 30pt

qa:54286
\begin{equation}
  y = f(x)= x^2+2x+1 + \int_{0}^{\pi} \frac{1+tx^2}{1-t^2}\,dt
  \vrule height 40pt depth 30pt
\end{equation}

下の実行例では添字のところで{\ybaselineshift0pt\cs{adjustbaseline}}未実行なため
補正幅がおかしくなっている
\[
\sqrt{\text{漢X\R}\R213}
\frac{\text{漢X\R}\R213}{s}
\sqrt{\hbox{漢X\R}}
\frac{\text{漢X\R}}{s}
  \displaystyle\int_{0}^{\text{漢X\R}\R1}
\vrule height 40pt depth 30pt
\]
\end{document}
